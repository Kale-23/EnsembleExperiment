% Options for packages loaded elsewhere
\PassOptionsToPackage{unicode}{hyperref}
\PassOptionsToPackage{hyphens}{url}
%
\documentclass[
]{article}
\usepackage{amsmath,amssymb}
\usepackage{lmodern}
\usepackage{iftex}
\ifPDFTeX
  \usepackage[T1]{fontenc}
  \usepackage[utf8]{inputenc}
  \usepackage{textcomp} % provide euro and other symbols
\else % if luatex or xetex
  \usepackage{unicode-math}
  \defaultfontfeatures{Scale=MatchLowercase}
  \defaultfontfeatures[\rmfamily]{Ligatures=TeX,Scale=1}
\fi
% Use upquote if available, for straight quotes in verbatim environments
\IfFileExists{upquote.sty}{\usepackage{upquote}}{}
\IfFileExists{microtype.sty}{% use microtype if available
  \usepackage[]{microtype}
  \UseMicrotypeSet[protrusion]{basicmath} % disable protrusion for tt fonts
}{}
\makeatletter
\@ifundefined{KOMAClassName}{% if non-KOMA class
  \IfFileExists{parskip.sty}{%
    \usepackage{parskip}
  }{% else
    \setlength{\parindent}{0pt}
    \setlength{\parskip}{6pt plus 2pt minus 1pt}}
}{% if KOMA class
  \KOMAoptions{parskip=half}}
\makeatother
\usepackage{xcolor}
\usepackage[margin=1in]{geometry}
\usepackage{graphicx}
\makeatletter
\def\maxwidth{\ifdim\Gin@nat@width>\linewidth\linewidth\else\Gin@nat@width\fi}
\def\maxheight{\ifdim\Gin@nat@height>\textheight\textheight\else\Gin@nat@height\fi}
\makeatother
% Scale images if necessary, so that they will not overflow the page
% margins by default, and it is still possible to overwrite the defaults
% using explicit options in \includegraphics[width, height, ...]{}
\setkeys{Gin}{width=\maxwidth,height=\maxheight,keepaspectratio}
% Set default figure placement to htbp
\makeatletter
\def\fps@figure{htbp}
\makeatother
\setlength{\emergencystretch}{3em} % prevent overfull lines
\providecommand{\tightlist}{%
  \setlength{\itemsep}{0pt}\setlength{\parskip}{0pt}}
\setcounter{secnumdepth}{-\maxdimen} % remove section numbering
\ifLuaTeX
  \usepackage{selnolig}  % disable illegal ligatures
\fi
\IfFileExists{bookmark.sty}{\usepackage{bookmark}}{\usepackage{hyperref}}
\IfFileExists{xurl.sty}{\usepackage{xurl}}{} % add URL line breaks if available
\urlstyle{same} % disable monospaced font for URLs
\hypersetup{
  hidelinks,
  pdfcreator={LaTeX via pandoc}}

\author{}
\date{\vspace{-2.5em}}

\begin{document}

\hypertarget{instructions-for-vr-experiment}{%
\section{Instructions for VR
Experiment}\label{instructions-for-vr-experiment}}

\begin{itemize}
\tightlist
\item
  if you have any questions, don't be afraid to text Kaleb for help @
  (603) 948 5346
\end{itemize}

\hypertarget{setup-before-participant-arrives}{%
\subsection{Setup Before Participant
Arrives}\label{setup-before-participant-arrives}}

\begin{itemize}
\tightlist
\item
  Make sure HDMI cable is connected to both the headset and computer
  through the GPU HDMI port

  \begin{itemize}
  \tightlist
  \item
    It should already be plugged in correctly, but if it is unplugged
    for any reason, text me and I can tell you how to plug it back in if
    you don't know how
  \end{itemize}
\item
  Plug in the VR headset power cable
\item
  Press the power button on the top of the computer and wait for the
  screen to startup
\item
  While starting up, run the headset through the CleanBox by pressing
  the green button in the top left

  \begin{itemize}
  \tightlist
  \item
    The button will turn red, and will turn back green when finished
  \end{itemize}
\item
  Once screen is on, select the lab profile and log in

  \begin{itemize}
  \tightlist
  \item
    user ID: svc\_UNH\_Psy\_Ksk01
  \item
    password: interdisciplinary perception lab

    \begin{itemize}
    \tightlist
    \item
      \textbf{(spaces included, no caps)}
    \end{itemize}
  \end{itemize}
\item
  Open the ``Vizard 7'' app, you will see the icon in the bottom bar
  \textbf{(seen below)}
\end{itemize}

\begin{itemize}
\tightlist
\item
  Check Sona system to make sure the participant is still signed up, and
  you have the right time.
\item
  Add participant to the ``Participant Logbook'' excel file with their
  appointment time and participant number.

  \begin{itemize}
  \tightlist
  \item
    The participant number can be found in the ``EnsembleExperiment''
    folder, and it is just the number seen in ``currentParticipant.txt''
  \item
    Keep the logbook open, and record when the participant arrives, when
    they put on the headset, and when they take it off.
  \end{itemize}
\item
  Within vizard, the experiment should already be opened and you should
  see ``Experiment.py'' open in the top left \textbf{(Seen below with a
  red box around it)}

  \begin{itemize}
  \tightlist
  \item
    If ``Experiment.py'' is not already open, locate the
    ``EnsembleExperiment'' folder on the desktop and drag the
    ``Experiment.py'' file into Vizard. You should then see it in the
    top left and it should be highlighted yellow to show it is open
  \end{itemize}
\end{itemize}

\begin{itemize}
\tightlist
\item
  Press the green arrow that is above and to the left of the
  ``Experiment.py'' label \textbf{(seen above with a red arrow pointing
  towards it)}

  \begin{itemize}
  \tightlist
  \item
    A new screen should pop up with a text entry field labeled ``Please
    put in your information''

    \begin{itemize}
    \tightlist
    \item
      During setup, skip putting in information and just click
      ``Submit'' right away, then press the space bar
    \end{itemize}
  \end{itemize}
\item
  Another screen called ``Mixed Reality Portal'' should open, and the
  headset should now connect. You can verify this by looking through the
  headset and making sure it is turned on.
\item
  Close out of the ``Experiment'' window.
\item
  If you do these steps too early, the headset will go into sleep mode
  and you will need to repeat all steps below the picture above.
\item
  Check the headset lenses, and clean them using the microfiber cloth if
  nessesary
\end{itemize}

\hypertarget{after-participant-arrives}{%
\subsection{After Participant Arrives}\label{after-participant-arrives}}

\begin{itemize}
\tightlist
\item
  Be on the lookout for the participant arriving, the hallway is
  confusing to people
\item
  When the participant arrives, greet them, confirm their name matches
  the name on the appointment, and bring them into the lab room.

  \begin{itemize}
  \tightlist
  \item
    On your way in, make sure to switch the sign to say ``Experiment in
    progress'' on the door
  \item
    Ask the participant to put all their devices on silent \textbf{(and
    you should do the same)} and place their phone/ smart watch/
    anything that could cause a distraction on the shelf or table nearby
  \end{itemize}
\item
  Ask the participant to sit down at a desk and hand them the consent
  form to fill out. Tell them if they have any questions about it or are
  confused that they can ask you for help or clarification.
\item
  Once they complete the consent form and you check to make sure
  everything is filled out correctly, sit the participant out front of
  the desk with the monitor.
\item
  \textbf{Put the consent form with all the others in a safe place}
\end{itemize}

\hypertarget{taking-ipd}{%
\subsubsection{Taking IPD}\label{taking-ipd}}

\begin{itemize}
\tightlist
\item
  Tell the participant you will now measure their inter pupulary
  distance, and you can explain what this is and why you need it if they
  want to know.
\item
  Warn the participant that you will get close and touch the bridge of
  their nose with a ruler
\item
  Once the participant consents to getting their IPD taken, use the
  ruler branded ``zyaid'' on the side to take their IDP

  \begin{itemize}
  \tightlist
  \item
    Ask the participant to stare directly over your shoulder, line the
    0mm mark up with the middle of their left pupil, and measure the
    distance to the middle of their right pupil. Make sure to write this
    number down or remember it for later.
  \item
    If needed, measure twice and then take the average for the IPD
  \end{itemize}
\end{itemize}

\hypertarget{starting-the-experiment}{%
\subsubsection{Starting the Experiment}\label{starting-the-experiment}}

\begin{itemize}
\item
  Once everything is ready, ask the participant to remove any head
  covering or hair style that may get in the way of the VR headset
  \textbf{(If they have a religious headcovering please don't ask them
  to remove it, try to work around it)}
\item
  Ask the participant which hand they would prefer to hold the
  controller in, and turn that one on. \textbf{(if both controllers are
  on, hold the windows button to turn the one not in use off)}
\item
  Start up the experiment again using the green arrow, and press the
  submit button then the space bar, close out of the window one last
  time, and then reopen it.
\item
  Have the participant fill out the form this time, making sure to put
  in the correct IPD from earlier.

  \begin{itemize}
  \tightlist
  \item
    When they are ready, they can press the submit button, and you can
    switch them to the VR headset
  \end{itemize}
\item
  Put the headset on the participant making sure to adjust the back knob
  to make it tight as well as the velcro top to pull it higher up
  \textbf{(if the headset is not properly adjusted, the participant will
  have blurry vision. Ask to make sure the headset is comfortable and
  that they can see clearly)}
\item
  Either ask to adjust the IPD using the slider below their right eye,
  or have the participant do it themselves. Get it as close as possible
  to the number you measured earlier
\item
  Hand the participant the controller, \textbf{make sure to slide the
  wrist wrap over their wrist, these controllers are expensive}
\item
  The participant will load into the waiting room in VR, and you can see
  this in the ``Mixed reality'' window

  \begin{itemize}
  \tightlist
  \item
    Teach the participant what buttons they need to press \textbf{(the
    trigger and joystick)}
  \item
    Teach the participant the windows and menu buttons and how to get
    rid of those if they pop up, but also that they should avoid
    pressing the buttons if possible.
  \end{itemize}
\item
  When the participant is ready to continue, you can let them know they
  will load into the experiment, and you can press the space button to
  continue
\item
  Look at the Vizard screen at the bottom, if you see the message below
  \textbf{(it will be highlighted in red in vizard)}, you need to
  restart the program by closing out and pressing the green arrow again.
  The controller will not work if this message is seen:
\end{itemize}

Traceback (most recent call last): ~ File
``C:\Program Files\WorldViz\Vizard7\python\viztask.py'', line 773, in
updateAndKillOnException ~~~~ return self.update() ~ File
``C:\Program Files\WorldViz\Vizard7\python\viztask.py'', line 738, in
update ~~~~ val = self.\_stack{[}-1{]}.send(sendData) ~ File
``C:\Users\svc\_UNH\_Psy\_Ksk01\Desktop\EnsembleExperiment\Experiment.py'',
line 525, in learningPhase ~~~~ yield viztask.waitSensorDown(controller,
{[}steamvr.BUTTON\_TRIGGER{]}) NameError: name `controller' is not
defined

\begin{itemize}
\tightlist
\item
  If this message does not show up, the experiment is ready to be run

  \begin{itemize}
  \tightlist
  \item
    If it does, make sure the correct controller is on, and then either
    ask the participant for their answers to the information panel, or
    have them take off the headset to redo it one last time.
  \end{itemize}
\end{itemize}

\hypertarget{guiding-participant-through-the-experiment}{%
\subsubsection{Guiding Participant Through the
Experiment}\label{guiding-participant-through-the-experiment}}

\begin{itemize}
\tightlist
\item
  When the participant begins, keep track of their progress using both
  the ``Mixed reality Portal'' and ``Experiment'' windows. Have one open
  on the left, and the other open on the right, as well as the
  ``Vizard'' window somewhere so you can read the bottom for any
  messages that appear.

  \begin{itemize}
  \tightlist
  \item
    You will see when they reach the tutorials through these windows, so
    ask them if they understand, and talk them through the learning
    phases. These are confusing the first time someone runs through
    them. Explain the goal, how to answer, and which buttons to use.

    \begin{itemize}
    \tightlist
    \item
      They have a learning phase for the depth test as well as the
      overall experiment, so make sure they seem to know what they are
      doing during these phases.
    \item
      \textbf{Do not guide them or give tips or any information through
      the actual test and experiment, we want as little instructor input
      after the learning phases so as not to skew the results. For
      example, do not tell them how many trials there are, or answer
      questions about how good they are doing. You can tell them ``I can
      answer that at the end of the experiment''}
    \end{itemize}
  \end{itemize}
\item
  You can let the participant know that they will have breaks around
  every 4-5 minutes, and that overall the experiment will take around 30
  minutes total.
\item
  During breaks, participants are allowed to take off their headset, but
  make sure it is no longer than 3-4 minutes so the controller and
  headset do not go to sleep.

  \begin{itemize}
  \tightlist
  \item
    Make sure at least one of the breaks are around 30 seconds so that
    the participant has time to rest.
  \end{itemize}
\item
  When the participant finishes, they will be told to take off their
  headset. Help the participant if they need it. If they reach this
  point, all the data should already be saved in the ``participantData''
  folder. It is safe to close all windows.

  \begin{itemize}
  \tightlist
  \item
    If another participant is scheduled, you can keep the ``Vizard'' and
    ``Mixed Reality Portal'' windows open, but need to close the
    ``Experiment'' window to reset the experiment and controller.
  \end{itemize}
\item
  Ask the participant if they have any questions about the experiment.
  Answer to the best of your knowledge. If they want to know more, you
  can give them my email:
  \href{mailto:kaleb.ducharme@unh.edu}{\nolinkurl{kaleb.ducharme@unh.edu}}.
\item
  Tell the participant that you will put their Sona credit into the
  system, and they should see it within a couple of minutes.
  \textbf{Make sure they grab all of their belongings before leaving,
  and to thank them for their time}
\item
  Put the rest of the information into the Excel Spreadsheet
  ``Participant LogBook''. \textbf{Make sure you check off that you
  provided Sona credit only after you already do so, so that we know the
  participant got their credit}
\item
  If you are the last person in lab, turn off the computer by pressing
  the windows logo at the bottom of the monitor, then the power button,
  and press ``Shut down''
\item
  Unplug the headset power cable from the connector
\item
  When leaving, make sure the keys are in the lock box \textbf{(dont
  want to lock them in the room)}, the lights are off, and that the door
  is locked
\end{itemize}

\end{document}
